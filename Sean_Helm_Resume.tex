\documentclass[10pt]{article}

\usepackage[margin=1.1cm]{geometry}
\usepackage[explicit]{titlesec}
\usepackage{enumitem}

\pagenumbering{gobble}
\setlength\parindent{0pt}
\setlist[itemize]{label=\textendash, noitemsep, topsep=3pt, leftmargin=*}

\newcommand*\ruleline[1]{\par\noindent\raisebox{.6ex}{\makebox[\linewidth]{\hrulefill\hspace{1ex}\raisebox{-.6ex}{#1}\hspace{1ex}\hrulefill}}}
\titleformat{\section}{\scshape\Large}{}{0pt}{\ruleline{#1}}
\titlespacing{\section}{0pt}{13pt}{0pt}

% Definitions for resume %

\def \normspace {4.5mm}

\newcommand{\nameinfo}[5]{
	\begin{center}
	{\Huge\textsc{#1}}\vspace{2mm}\break
	\begin{tabular}{@{}c|c|c|c}
	#2 & #3 & #4 & #5
	\end{tabular}
	\end{center}
}

\newcommand{\job}[3]{
	\vspace{\normspace}
	\makebox[\textwidth]{{\large \textbf{#1}}\hfill \textbf{#3}}\vspace{1mm}
	{\large \textsl{#2}}
}

\newcommand{\project}[1]{
	\vspace{\normspace}
	{\large \textbf{#1}}
}

\newcommand{\school}[3]{
	\vspace{\normspace}
	\makebox[\textwidth]{{\large \textbf{#1}}\hfill \textbf{#3}}\vspace{1mm}
	{\large \textsl{#2}}
}

% Resume %

\begin{document}

\nameinfo{Sean Helm}{\textbf{seanhelm}.com}{\textbf{seanmdhelm}@gmail.com}{github.com/\textbf{seanhelm}}{linkedin.com/in/\textbf{seanhelm}}

\section*{Experience}

\job{Capital One}{Data/Software Engineer}{Sep 2018 --}
\begin{itemize}
\item Architected a caching solution in Java/Spring for a big data platform to reduce latency for recently run Snowflake queries.
\item Lowered operational costs by enabling lifecycle management for the underlying cache using Python and AWS Lambda.
\item Captured clickstream data with React.js and relayed to Amazon Redshift for user workflow analysis.
\item Built out infrastructure with Jenkins and Terraform to automate deployment of AWS resources for the cache microservice.
\end{itemize}

\job{Alarm.com}{Software Engineering Intern}{Jun 2017 -- Aug 2017}
\begin{itemize}
\item Optimized home automation testing by implementing a Windows service in C\# for allocating device tests to available servers.
\item Leveraged Ember.js to provide a user interface for the distributed testing service and to display real-time testing results.
\item Created a REST API with ASP.NET Web API to expose the distributed testing service to the client.
\end{itemize}

\job{Alarm.com}{Software Engineering Intern}{Jun 2016 -- Aug 2016}
\begin{itemize}
\item Automated a tedious process by building an API in C\# to efficiently update branding information on security panels.
\item Implemented a caching system for account information to speed up the dealer site page load speed by 40-60\%.
\item Developed a responsive, customer-facing tool to allow home owners to search for and verify local home security dealers.
\end{itemize}

\job{Sure Secure Solutions}{Software Engineering Intern}{Aug 2014 -- Aug 2015}
\begin{itemize}
\item Updated a static archive to a secure and dynamic web portal in PHP for easier management.
\item Redesigned the company website using HTML/CSS and JavaScript to make web pages robust and responsive.
\end{itemize}

\section*{Education}

\school{George Mason University}{Bachelor of Science in Computer Science; Major GPA: 3.38.}{May 2018}

\section*{Projects}

\project{Music Genre Classification with Gradient Boosting}
\begin{itemize}
\item Applied XGBoost machine learning algorithm in Python to classify music genres using features extracted with LibROSA.
\item Utilized Matplotlib to produce visualizations of various performance metrics to help improve the classification model.
\end{itemize}

\project{Reddit Sentiment Analysis}
\begin{itemize}
\item Filtered negative Reddit comments in real-time into MongoDB using TextBlob sentiment analysis.
\item Created a user-friendly web application using Node.js, Express, and Vue.js to review newly extracted comments.
\end{itemize}

\project{Toxic Mushroom Deep Dive}
\begin{itemize}
\item Compared Scikit-learn machine learning algorithms in Python for predicting whether mushrooms were toxic or edible.
\end{itemize}

\section*{Certifications}

\school{Certified Solutions Architect, Associate}{Amazon Web Services; Expiration Date: Dec 2020}{Dec 2018}

\end{document}